\documentclass[10pt, a4paper, oneside]{basestyle}
\usepackage[Mathematics]{semtex}
\usepackage{chngcntr}
\counterwithout{equation}{section}

%%%% Shorthands.

%%%% Title and authors.

\title{%
\textdisplay{%
Geopotential%
}%
}
\author{Pascal~Leroy (phl)}
\begin{document}
\maketitle
\noindent
This document describes the computations that are performed by the method \texttt{GeneralSphericalHarmonicsAcceleration} of class \texttt{Geopotential} to the determine the acceleration exerted by a non-spherical celestial on a point mass.

Let $\vr$ be the vector going from the centre of the celestial to the point mass.  Let $\tuple{\vx, \vy, \vz}$ be a (direct) base whose $\vz$ axis is along the axis of rotation of the celestial and whose $\vx$ axis points toward a reference point on the celestial.  In this base $\vr$ has coordinates $\tuple{x, y, z}$ which can be expressed in terms of the latitude $\gb$ and the longitude $\gl$:
\begin{equation*}
\vr=\begin{pmatrix}
x \\ y \\ z
\end{pmatrix}
=\begin{pmatrix}
r \cos \gb \cos \gl\\
r \cos \gb \sin \gl\\
r \sin \gb
\end{pmatrix}
\end{equation*}
where $r$ is the norm of $\vr$.

The gravitational potential exerted by the celestial has the form:
\begin{equation*}
U\of\vr = -\frac{\gm}{r}\pa{1 + \sum{n = 1}[\infty] \sum{m = 0}[n] 
\pa{\frac{R}{r}}^n P_{m n}\of{\sin \gb}
\pa{C_{m n} \cos m \gl + S_{m n} \sin m \gl}
}
\end{equation*}
where $P_{m n}$ is the associated Legendre function defined as:
\begin{equation*}
P_{m n} = \pa{1 - t^2}^{\frac{m}{2}} \deriv[m] t {P_n\of t}
\end{equation*}
The terms that are relevant for evaluating the acceleration exerted by the spherical harmonics on the point mass have the form:
\begin{equation*}
V_{n m}\of\vr = \frac{1}{r}\pa{\frac{R}{r}}^n P_{m n}\of{\sin \gb}
\pa{C_{m n} \cos m \gl + S_{m n} \sin m \gl}
\end{equation*}
and the acceleration itself is proportional to $\grad V_{n m}\of\vr$.

$V_{n m}\of\vr$ can be written as the product of three terms dependent of $r$, $\gb$, and $\gl$, respectively:
\begin{equation*}
V_{n m}\of\vr = {\mathfrak R\of r} {\mathfrak B\of \gb} {\mathfrak L\of \gl}
\end{equation*}
where:
\begin{equation*}
\begin{cases}
\mathfrak R\of r &= \frac{1}{r}\pa{\frac{R}{r}}^n\\
\mathfrak B\of \gb &= P_{m n}\of{\sin \gb}\\
\mathfrak L\of \gl &= C_{m n} \cos m \gl + S_{m n} \sin m \gl
\end{cases}
\end{equation*}
\end{document}