\documentclass[10pt, a4paper, oneside]{basestyle}
\usepackage[Mathematics]{semtex}

%%%% Shorthands.

%%%% Title and authors.

\title{%
\textdisplay{%
Calculations for the second-degree zonal harmonic%
}%
}
\author{Pascal~Leroy (phl) \& Robin~Leroy (eggrobin)}
\begin{document}
\maketitle
\noindent
Notations:\[
\vr=\begin{pmatrix}
x \\ y \\ z
\end{pmatrix} \text{, }
\norm{\vz}=1 \text{, }
r = \norm{\vr}\text.
\]
For oblateness along the $z$-axis, the potential is
\[
\frac{J_2}{2r^5}\pa{3z^2 - r^2}\text.
\]
For oblateness along $\vz$,
\[
U\of\vr = \frac{J_2}{2r^5}\pa{3\pascal{\scal{\vr}{\vz}}^2 - r^2}\text.
\]
Differentiating,
\[
\deriv \vr U = \frac{J_2}{2}\pa{-\frac{5}{r^6} \deriv \vr r
                   \pa{3\pascal{\scal\vr\vz}^2 - r^2}
               + \frac{1}{r^5}\pa{3\derivop \vr \pascal{\scal\vr\vz}^2 -
                   2 r \deriv \vr r}}\text.
\]
Recall that $\deriv \vr r = \frac{\vr}{r}$,
\begin{align*}
\deriv \vr U &= \frac{J_2}{2}\pa{-\frac{5\vr}{r^7}
                    \pa{3\pascal{\scal\vr\vz}^2 - r^2}
                + \frac{1}{r^5}\pa{3\derivop \vr \pascal{\scal\vr\vz}^2 -
                    2 \vr}} \\
             &= \frac{J_2}{2}\pa{-\frac{15\vr}{r^7}
                    \pascal{\scal\vr\vz}^2 + \frac{3\vr}{r^5}
                + \frac{3}{r^5}3\derivop \vr \pascal{\scal\vr\vz}^2}\text.
\end{align*}
With $\derivop \vr \pascal{\scal\vr\vz}^2 =
  2\pascal{\scal\vr\vz}\deriv\vr{\pascal{\scal\vr\vz}} = 2\pascal{\scal\vr\vz}\vz$,
\begin{align*}
\deriv \vr U &= \frac{J_2}{2}\pa{-\frac{15\vr}{r^7}
                    \pascal{\scal\vr\vz}^2 + \frac{3\vr}{r^5}
                + \frac{6\vz}{r^5}\pascal{\scal\vr\vz}} \\
             &= \frac{3J_2}{2r^5}\pa{2\vz\pascal{\scal\vr\vz}
                + \vr \pa{1 - \frac{5\pascal{\scal\vr\vz}^2}{r^2}}}\text.
\end{align*}
Note that this is invariant under $\vz\mapsto -\vz$ (but not under $\vr\mapsto -\vr$).


\end{document}