\documentclass[10pt, a4paper, twoside]{basestyle}

\usepackage[backend=biber,firstinits=true,maxnames=100,style=alphabetic,maxalphanames=4,doi=true,isbn=false,url=false,eprint=true]{biblatex}
\bibliography{bibliography}

\usepackage[Mathematics]{semtex}
\usepackage{chngcntr}
\counterwithout{equation}{section}

%%%% Shorthands.

%%%% Title and authors.

\title{%
\textdisplay{%
On an Article by Celledoni et al.%
}%
}
\author{Pascal~Leroy (phl)}
\begin{document}
\maketitle
\begin{sloppypar}
\noindent
This document provides clarifications, corrections, and accuracy improvements to the formul{\ae} presented in \cite{Celledoni2008}.  It follows the notation
and conventions of that paper.
\end{sloppypar}

\section*{Preamble}
We remind the reader of the derivation formul{\ae} for the Jacobian elliptic functions (\cite{NistHMF2010}, section 22.13(i)):
\[
\begin{dcases}
\derivop{u}{\JacobiSN u} &= \JacobiCN u \JacobiDN u \\
\derivop{u}{\JacobiCN u} &= -\JacobiSN u \JacobiDN u \\
\derivop{u}{\JacobiDN u} &= -k^2 \JacobiSN u \JacobiCN u
\end{dcases}
\]
and for the hyperbolic functions (\cite{NistHMF2010}, section 4.34):
\[
\begin{dcases}
\derivop{u}{\HyperbolicTangent u} &= \HyperbolicSecant^2 u \\
\derivop{u}{\HyperbolicSecant u} &= -\HyperbolicSecant u \HyperbolicTangent u
\end{dcases}
\]

\section*{The equations of motion}
We start by writing equation (1) of \cite{Celledoni2008} in coordinates.  The coordinates of $\vm$ and $\VectorSymbol{I}$ are defined by:
\[
\vm\DefineAs
\begin{pmatrix}
m_1 \\ m_2 \\ m_3
\end{pmatrix}
\]
and:
\[
\VectorSymbol{I}\DefineAs
\begin{pmatrix}
I_1 & 0 & 0 \\ 0 & I_2 & 0 \\ 0 & 0 & I_3
\end{pmatrix}
\]
with $I_1 \leq I_2 \leq I_3$.

Euler's equation $\TimeDerivative{\vm} = \commutator{m}{\gw}$ can be written in coordinates in the principal axes frame:
\[
\TimeDerivative{\vm} =
\begin{pmatrix}
m_1 \\ m_2 \\ m_3
\end{pmatrix}
\times
\begin{pmatrix}
m_1/I_1 \\ m_2/I_2 \\ m_3/I_3
\end{pmatrix}
\]
thus:
\begin{equation}
\begin{dcases}
\TimeDerivative{m}_1 &= m_2 m_3 \pa{1/I_3 - 1/I_2}\\
\TimeDerivative{m}_2 &= m_3 m_1 \pa{1/I_1 - 1/I_3}\\
\TimeDerivative{m}_3 &= m_1 m_2 \pa{1/I_2 - 1/I_1}
\end{dcases}
\label{eqneuler}
\end{equation}
\subsection*{Solution of Euler's equation, case (i)}
Case (i) of the solution of Euler's equation in section 2.2 of \cite{Celledoni2008} is:
\[
{\vm}_t =
\begin{pmatrix}
\gs B_{13} \JacobiDN\of{\gl t - \gn, k} \\
-B_{21} \JacobiSN\of{\gl t - \gn, k} \\
B_{31} \JacobiCN\of{\gl t - \gn, k}
\end{pmatrix}
\]
If we derive this expression with respect to $t$, inject in into (\ref{eqneuler}), and eliminate the elliptic functions we obtain:
\begin{equation}
\begin{dcases}
-\gs \gl k^2 B_{13} &= -B_{21} B_{31} \pa{1/I_3 - 1/I_2} \\
-\gl B_{21} &= \gs B_{13} B_{31} \pa{1/I_1 - 1/I_3} \\
-\gl B_{31} &= -\gs B_{13} B_{21} \pa{1/I_2 - 1/I_1}
\end{dcases}
\label{solneuleri}
\end{equation}
The last equation of (\ref{solneuleri}) yields the following value for $\gl$:
\begin{align*}
\gl &= \gs \frac{B_{13} B_{21}}{B_{31}} \frac{I_1 - I_2}{I_1 I_2}
= \gs\sqrt{\frac{I_1 \gD_3}{I_{13}} \frac{I_2 \gD_1}{I_{21}} \frac{I_{31}}{I_3 \gD_1}} \frac{I_1 - I_2}{I_1 I_2} \\
&= \gs\sqrt{\frac{\gD_3}{I_{21} I_1 I_2 I_3}} \pa{I_1 - I_2}
= -\gs\sqrt{\frac{\gD_3 I_{21}}{I_1 I_2 I_3}}
= -\gs \gl_3
\end{align*}
The sign change when moving $I_1 - I_2$ under the radical is necessary because $I_1 - I_2 < 0$.

It is straightforward to check that this value of $\gl$ also satisfies the other equations of (\ref{solneuleri}).  Note that it
differs in sign from the one given by \cite{Celledoni2008}: the sign error is visible in that it does not yield the proper precession 
direction.

\subsection*{Solution of Euler's equation, case (ii)}
Case (ii) of the solution of Euler's equation in section 2.2 of \cite{Celledoni2008} is:
\[
{\vm}_t =
\begin{pmatrix}
B_{13} \JacobiCN\of{\gl t - \gn, k^{-1}} \\
-B_{23} \JacobiSN\of{\gl t - \gn, k^{-1}} \\
\gs B_{31} \JacobiDN\of{\gl t - \gn, k^{-1}}
\end{pmatrix}
\]
Just as we did above, we derive this expression with respect to $t$, inject in into (\ref{eqneuler}), and eliminate the elliptic functions:
\begin{equation}
\begin{dcases}
-\gl B_{13} &= -\gs B_{23} B_{31} \pa{1/I_3 - 1/I_2} \\
-\gl B_{23} &= \gs B_{13} B_{31} \pa{1/I_1 - 1/I_3} \\
-\gs \gl k^{-2} B_{31} &= -B_{13} B_{23} \pa{1/I_2 - 1/I_1}
\end{dcases}
\label{solneulerii}
\end{equation}
The first equation of (\ref{solneulerii}) yields the following value for $\gl$:
\begin{align*}
\gl &= \gs \frac{B_{23} B_{31}}{B_{13}} \frac{I_2 - I_3}{I_2 I_3}
= \gs\sqrt{\frac{I_2 \gD_3}{I_{23}} \frac{I_3 \gD_1}{I_{31}} \frac{I_{13}}{I_1 \gD_3}} \frac{I_2 - I_3}{I_2 I_3} \\
&= \gs\sqrt{\frac{\gD_1}{I_{23} I_1 I_2 I_3}} \pa{I_2 - I_3}
= -\gs\sqrt{\frac{\gD_1 I_{23}}{I_1 I_2 I_3}}
= -\gs \gl_1
\end{align*}
Again, note the change of sign due to the fact that $I_2 - I_3 < 0$.  And again, the same value of $\gl$ can be shown to satisfy the other
equations of (\ref{solneulerii}).

\subsection*{Solution of Euler's equation, case (iii)}
Case (iii) of the solution of Euler's equation in section 2.2 of \cite{Celledoni2008} is clearly incorrect as it implies that $m_1$ and $m_3$
always have the same sign, whereas it is straightforward to choose initial conditions where they do not.  Instead, we introduce an extra parameter
$\gs'' = ±1$ and posit a solution of the form:
\[
{\vm}_t =
\begin{pmatrix}
\gs' B_{13} \HyperbolicSecant\of{\gl t - \gn} \\
\HyperbolicTangent\of{\gl t - \gn} \\
\gs'' B_{31} \HyperbolicSecant\of{\gl t - \gn}
\end{pmatrix}
\]
Deriving this expression and injecting it into (\ref{eqneuler}) yields:
\begin{equation}
\begin{dcases}
-\gs' \gl B_{13} &= \gs'' B_{31} \pa{1/I_3 - 1/I_2} \\
\gl &= \gs' \gs'' B_{13} B_{31} \pa{1/I_1 - 1/I_3} \\
-\gs'' \gl B_{31} &= \gs' B_{13} \pa{1/I_2 - 1/I_1}
\end{dcases}
\label{solneuleriii}
\end{equation}
The second equation of (\ref{solneuleriii}) gives the following value for $\gl$:
\[
\gl = \gs' \gs'' B_{13} B_{31} \frac{I_3 - I_1}{I_1 I_3}
= \gs' \gs'' \sqrt{\frac{I_1 \gD_3}{I_{13}} \frac{I_3 \gD_1}{I_{31}}} \frac{I_3 - I_1}{I_1 I_3}
= \gs' \gs'' \sqrt{\frac{\gD_1 \gD_3}{I_1 I_3}}
\]
In this case it is a bit less obvious that the other equations yield the same value of $\gl$.  We detail the derivation for the first equation,
using the fact that ${\gs'}^2 = 1$:
\begin{align*}
\gl &= -\gs' \gs'' \frac{B_{31}}{B_{13}} \frac{I_2 - I_3}{I_2 I_3}
= -\gs' \gs'' \sqrt{\frac{I_3 \gD_1}{I_{31}} \frac{I_{13}}{I_1 \gD_3}} \frac{I_2 - I_3}{I_2 I_3} \\
&= -\gs' \gs'' \sqrt{\frac{\gD_1}{I_1 I_3 \gD_3}} \frac{I_2 - I_3}{I_2}
= \gs' \gs'' \sqrt{\frac{\gD_1}{I_1 I_3 \gD_3}} \pa{\frac{I_3}{I_2} - 1}
\end{align*}
Now note that in case (iii) we have $2 T I_2 = 1$ thus $1/I_2 = 2 T$.  $\gl$ can be rewritten as:
\[
\gl = \gs' \gs'' \sqrt{\frac{\gD_1}{I_1 I_3 \gD_3}} \pa{2 T I_3 - 1} = \gs' \gs'' \sqrt{\frac{\gD_1 \gD_3}{I_1 I_3}}
\]
where we have used the fact that $2 T I_3 - 1 = 2 T \pa{I_3 - I_2} > 0$.

We then define:
\[
\gl_2 = \sqrt{\frac{\gD_1 \gD_3}{I_1 I_3}}
\]
It is easy to see that $\gl_2$ is the common value of $\gl_1$ and $\gl_3$ in case (iii), that $\gs'$ and $\gs''$ are free parameters and that:
\[
\gl = \gs' \gs'' \gl_2
\]

\subsection*{Phase and initial value}
The phase $\gn$ and the free parameters $\gs$, $\gs'$ and $\gs''$ are determined from the initial value ${\vm}_0$ by setting $t = 0$.
\subsubsection*{Case (i)}
We have:
\[
{\vm}_0 =
\begin{pmatrix}
\gs B_{13} \JacobiDN\of{-\gn, k} \\
-B_{21} \JacobiSN\of{-\gn, k} \\
B_{31} \JacobiCN\of{-\gn, k}
\end{pmatrix}
\]
First, we set $\gs$ to be the sign of $m_{01}$.  Then, forming the quotient of the last two coordinates we find:
\[
\frac{m_{02}}{m_{03}} = \frac{B_{21}}{B_{31}}\TrigonometricTangent\of{\JacobiAmplitude\of{\gn, k}}
\]
thus:
\[
\InverseTrigonometricTangent\of{\frac{m_{02}}{m_{03}} \frac{B_{31}}{B_{21}}} = \JacobiAmplitude\of{\gn, k}
\]
and finally we obtain $\gn$ as:
\[
\gn = F\of{\InverseTrigonometricTangent\of{\frac{m_{02}}{m_{03}} \frac{B_{31}}{B_{21}}}, k}
\]

\subsubsection*{Case (ii)}
Starting from:
\[
{\vm}_0 =
\begin{pmatrix}
B_{13} \JacobiCN\of{-\gn, k^{-1}} \\
-B_{23} \JacobiSN\of{-\gn, k^{-1}} \\
\gs B_{31} \JacobiDN\of{-\gn, k^{-1}}
\end{pmatrix}
\]
we set $\gs$ to be the sign of $m_{03}$ and form the quotient of the first two coordinates.  We obtain:
\[
\frac{m_{02}}{m_{01}} = \frac{B_{23}}{B_{13}} \TrigonometricTangent\of{\JacobiAmplitude\of{\gn, k^{-1}}} 
\]
and for $\gn$:
\[
\gn = F\of{\InverseTrigonometricTangent\of{\frac{m_{02}}{m_{01}} \frac{B_{13}}{B_{23}}}, k^{-1}}
\]

\subsubsection*{Case (iii)}
The initial value ${\vm}_0$  is:
\[
{\vm}_0 =
\begin{pmatrix}
\gs' B_{13} \HyperbolicSecant\of{-\gn} \\
\HyperbolicTangent\of{-\gn} \\
\gs'' B_{31} \HyperbolicSecant\of{-\gn}
\end{pmatrix}
\]
$\gs'$ and $\gs''$ are set to be the signs of $m_{01}$ and $m_{03}$, respectively.  The second coordinate immediately gives:
\[
\gn = -\InverseHyperbolicTangent\of{m_{02}}
\]

\subsection*{Implementation considerations}
Some of the formul{\ae} given by \cite{Celledoni2008} do not lend themselves to an easy implementation or lead to numerical inaccuracies.  We
describe in this section the modifications we make to these formul{\ae} in our implementation.  We also restore dimensionful formul{\ae} as needed.

\subsubsection*{The quantity $I_{jh}$}
It is simpler and more efficient to avoid absolute values, so we define:
\[
I_{jh} \DefineAs I_j - I_h
\]
This is the same quantity as in \cite{Celledoni2008} when $j \geq h$ but it has the opposite sign (it is negative) when $j < h$.

\subsubsection*{The quantity $\gD_j$}
We notice that the computation of $\gD_j$ may entail cancellations, so we go back to the definition of $\norm{\vm}$ and of the kinetic energy:
\[
\begin{dcases}
G^2 &= m_1^2 + m_2^2 + m_3^2 \\
2 T &= \frac{m_1^2}{I_1} + \frac{m_2^2}{I_2} + \frac{m_3^2}{I_3}
\end{dcases}
\]
and we define a dimensionful $\gD_j$ without absolute values:
\[
\gD_j \DefineAs G^2 - 2 T I_j
\]
When, for instance, $j = 2$, this yields:
\begin{align*}
\gD_2 &= m_1^2 \pa{1 - \frac{I_2}{I_1}} + m_3^2 \pa{1 - \frac{I_2}{I_3}} \\
&= m_1^2 \frac{I_{12}}{I_1} + m_3^2 \frac{I_{32}}{I_3}
\end{align*}
and similarly:
\[
\begin{dcases}
\gD_1 &= m_2^2 \frac{I_{21}}{I_2} + m_3^2 \frac{I_{31}}{I_3} \\
\gD_3 &= m_1^2 \frac{I_{13}}{I_1} + m_2^2 \frac{I_{23}}{I_2}
\end{dcases}
\]
It is easy to see that $\gD_1$ and $\gD_3$ are the sums of terms of the same sign, so they can be computed without cancellations.  Furthermore,
$\gD_1 \geq 0$ and $\gD_3 \leq 0$.  $\gD_2$ can have either sign, which correspond exactly to cases (i) ($\gD_2 < 0$), (ii) ($\gD_2 > 0$) and 
(iii) ($\gD_2 = 0$).

\subsubsection*{The elliptic modulus}
For the computation of the elliptic functions and integrals \cite{Celledoni2008} gives the value of the elliptic modulus $k$ but we need the value of
the complementary parameter $m_c = 1 - m$ (see \cite{NistHMF2010}, section 19.1.2 for an overview of the notation).  In case (i) we have:
\[
m_c = 1 - k^2 = 1 + \frac{\gD_1 I_{32}}{\gD_3 I_{21}}
\]
where we have used $\gD_3 \leq 0$.  This can be rewritten as follows:
\begin{align*}
m_c &= \frac{\gD_3 I_{21} + \gD_1 I_{32}}{\gD_3 I_{21}}
=\frac{\pa{G^2 - 2 T I_3}\pa{I_2 - I_1} + \pa{G^2 - 2 T I_1}\pa{I_3 - I_2}}{\gD_3 I_{21}} \\
&=\frac{G^2\pa{I_3 - I_1} + 2 T I_2\pa{I_1 - I_3}}{\gD_3 I_{21}}
=\frac{\gD_2 I_{31}}{\gD_3 I_{21}}
\end{align*}

Similarly, in case (ii):
\[
m_c = 1 - k^{-2} = 1 + \frac{\gD_3 I_{21}}{\gD_1 I_{32}}
=\frac{\gD_1 I_{32} + \gD_3 I_{21}}{\gD_1 I_{32}}
=\frac{\gD_2 I_{31}}{\gD_1 I_{32}}
\]
Note that in both cases we have $m_c \geq 0$.

\subsubsection*{The dimensionful equations}
$I_{jh}$ has the physical dimension of an inertial momentum $L^2 M$.  $G$ has the dimension of an angular momentum $L^2 M T^{-1} A$.  $\gD_j$ has 
the same dimension as $G^2$.  $B_{jh}$ has the same dimension as $\sqrt{\gD_h}$, i.e., the same dimension as $G$.  $\gl_1$ and $\gl_3$ have the
same dimension as the quotient $\frac{G}{I_j}$, i.e., $T^{-1} A$ which is appropriate for their usage.

This analysis shows that most of the formul{\ae} in \cite{Celledoni2008} have proper physical dimensions, expect for the $1$ in the definition of
$\gD_j$ (which should be $G^2$) and for the second coordinate of $\vm$ in case (iii) which should be:
\[
m_{2t} = G \HyperbolicTangent\of{-\gn}
\]
With this adjustment we obtain the dimensionful formul{\ae}:
\begin{align*}
\gn &= -\InverseHyperbolicTangent\of{\frac{m_{02}}{G}} \\
\gl_2 &= \frac{1}{G} \sqrt{\frac{\gD_1 \gD_3}{I_1 I_3}}
\end{align*}
It is easy to verify that the argument of the $\InverseHyperbolicTangent$ is dimensionless and that $\gl_2$ has the same dimension as the quotient 
$\frac{\gD_j}{G I_j}$, which has the same dimension as $\frac{G}{I_j}$, namely, $T^{-1} A$.
\printbibliography
\end{document}