\documentclass[10pt, a4paper, twoside]{basestyle}

\usepackage[backend=biber,firstinits=true,maxnames=100,style=alphabetic,maxalphanames=4,doi=true,isbn=false,url=false,eprint=true]{biblatex}
\bibliography{bibliography}

\usepackage[Mathematics]{semtex}
\usepackage{chngcntr}
\counterwithout{equation}{section}

%%%% Shorthands.

%%%% Title and authors.

\title{%
\textdisplay{%
On an Article by Celledoni et al.%
}%
}
\author{Pascal~Leroy (phl)}
\begin{document}
\maketitle
\begin{sloppypar}
\noindent
This document provides clarifications, corrections, and accuracy improvements to the formul{\ae} presented in \cite{Celledoni2008}.  It follows the notation
and conventions of that paper.
\end{sloppypar}

\section*{Preamble}
We remind the reader of the derivation formul{\ae} for the Jacobian elliptic functions (\cite{NistHMF2010}, section 22.13(i)):
\[
\begin{dcases}
\derivop{u}{\JacobiSN u} &= \JacobiCN u \JacobiDN u \\
\derivop{u}{\JacobiCN u} &= -\JacobiSN u \JacobiDN u \\
\derivop{u}{\JacobiDN u} &= -k^2 \JacobiSN u \JacobiCN u
\end{dcases}
\]

\section*{The equations of motion}
We start by writing equation (1) of \cite{Celledoni2008} in coordinates.  The coordinates of $m$ and $I$ are defined by:
\[
\vm\DefineAs
\begin{pmatrix}
m_1 \\ m_2 \\ m_3
\end{pmatrix}
\]
and:
\[
\VectorSymbol{I}\DefineAs
\begin{pmatrix}
I_1 & 0 & 0 \\ 0 & I_2 & 0 \\ 0 & 0 & I_3
\end{pmatrix}
\]
Euler's equation $\TimeDerivative{m} = m\Exterior\pa{I^{-1} m}$ can be written in coordinates:
\[
\TimeDerivative{\vm} =
\begin{pmatrix}
m_1 \\ m_2 \\ m_3
\end{pmatrix}
\Exterior
\begin{pmatrix}
m_1/I_1 \\ m_2/I_2 \\ m_3/I_3
\end{pmatrix}
\]
thus:
\begin{equation}
\begin{dcases}
\TimeDerivative{m}_1 &= m_2 m_3 \pa{1/I_3 - 1/I_2}\\
\TimeDerivative{m}_2 &= m_3 m_1 \pa{1/I_1 - 1/I_3}\\
\TimeDerivative{m}_3 &= m_1 m_2 \pa{1/I_2 - 1/I_1}
\end{dcases}
\label{eqneuler}
\end{equation}
\subsection*{Solution of Euler's equation, case (i)}
The case (i) of the solution of Euler's equation in section 2.2 of \cite{Celledoni2008} is:
\[
{\vm}_t =
\begin{pmatrix}
\gs B_{13} \JacobiDN\of{\gl t - \gn, k} \\
-B_{21} \JacobiSN\of{\gl t - \gn, k} \\
B_{31} \JacobiCN\of{\gl t - \gn, k}
\end{pmatrix}
\]
If we derive this expression with respect to $t$, inject in into (\ref{eqneuler}), and eliminate the elliptic functions we obtain:
\begin{equation}
\begin{dcases}
-\gs \gl k^2 B_{13} &= -B_{21} B_{31} \pa{1/I_3 - 1/I_2} \\
-\gl B_{21} &= \gs B_{13} B_{31} \pa{1/I_1 - 1/I_3} \\
-\gl B_{31} &= -\gs B_{13} B_{21} \pa{1/I_2 - 1/I_1}
\end{dcases}
\label{solneuleri}
\end{equation}
The last equation of (\ref{solneuleri}) yields the following value for $\gl$:
\[
\gl = \gs \frac{B_{13} B_{21}}{B_{31}}\frac{I_1 - I_2}{I_1 I_2}
= \gs\sqrt{\frac{I_1 \gD_3}{I_{13}} \frac{I_2 \gD_1}{I_{21}} \frac{I_{31}}{I_3 \gD_1}} \frac{I_1 - I_2}{I_1 I_2}
= \gs\sqrt{\frac{\gD_3}{I_{21} I_1 I_2 I_3}} \pa{I_1 - I_2}
= -\gs\sqrt{\frac{\gD_3 I_{21}}{I_1 I_2 I_3}}
= -\gs \gl_3
\]
The sign change when moving $I_1 - I_2$ under the radical is necessary because $I_1 - I_2 < 0$.

It is straightforward to check that this value of $\gl$ also satisfies the other equations of (\ref{solneuleri}).  Note that it
differs in sign from the one given by \cite{Celledoni2008}: the sign error is visible in that it does not yield the proper precession 
direction.
\printbibliography
\end{document}